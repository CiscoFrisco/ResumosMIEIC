\documentclass[../resumosRCOM.tex]{subfiles}
 
\begin{document} 
\subsection{Transmitting information}

\begin{equation}
    C = 2B\log_2 M
\end{equation}

\begin{description}
    \item[C] \hfill \\ channel capacity
    \item[B] \hfill \\ bandwith
    \item[2B] \hfill \\ baudrate in symbol/s or baud
    \item[M] \hfill \\ levels used to encode information  
\end{description}

\subsection{Types of modulations}

\begin{itemize}
    \item Binary signal
    \item Amplitude modulation
    \item Frequency modulation
    \item Phase modulation
    \item Quadrature Amplitude Modulation (M - QAM)
\end{itemize}

\subsection{Shannon's Law}

The maximum theoretical capacity of a channel (bit/s) is given by the following expressions:
\begin{equation}
    SNR = \frac{P_r}{N_0B_c}
\end{equation}

\begin{equation}
    C = B_c\log_2 (1 + SNR)
\end{equation}

\begin{description}
    \item[SNR] \hfill \\ signal to noise ratio
    \item[\(B_c\)] \hfill \\ bandwidth of the channel (Hz)
    \item[\(P_r\)] \hfill \\  signal power as seen by receiver (W) 
    \item[\(N_0\)] \hfill \\  White noise; noise power per unit bandwidth (W/Hz)
\end{description}

\subsection{Free space loss}

\begin{equation}
    \frac{P_t}{P_r}=\frac{(4\lambda fd)^2}{c^2}
\end{equation}

\begin{description}
    \item[\(P_t\)] \hfill \\ signal power at transmitting antenna
    \item[\(P_r\)] \hfill \\ signal power at receiving antenna
    \item[\(\lambda\)] \hfill \\ carrier wavelength
    \item[d] \hfill \\ propagation distance between antennas
    \item[c] \hfill \\ speed of light \(3 * 10^8 \) m/s 
\end{description}

\subsection{Solved Exam Problems}
\paragraph{2018R - 1}
\begin{itemize}
    \item 16 QAM
    \item bitrate (C) = 8kbit/s
    \item baudrate (2B) = ?
\end{itemize}

\[C = 2B\log_2 M\]
\[8 = 2B\log_2 16\]
\[8 = 2B*4\]
\[2B = 2\]

\paragraph{2018N - 1}
\begin{itemize}
    \item 8PSK
    \item baudrate (2B) = 250 kbaud
    \item bitrate (C) = ?
\end{itemize}

\[C = 2B\log_2 M\]
\[C = 250\log_2 8\]
\[C = 250*3\]
\[C = 750\]

\paragraph{2017N - 2}
\begin{itemize}
    \item baudrate (2B) = 100 kbaud
    \item bitrate (C) = 300 kbit/s
    \item phase modulation
    \item nº of phases = ?
\end{itemize}

\[C = 2B\log_2 M\]
\[300 = 100\log_2 M\]
\[3 = log_2 M\]
\[M = 2^3 = 8\]

\paragraph{2017N - 3}
\[\frac{P_t}{P_r}=\frac{(4\lambda fd)^2}{c^2}\]
\[{P_r}=\frac{P_t}{\frac{(4\lambda fd)^2}{c^2}}\]
\[{P_r}=\frac{P_tc^2}{(4\lambda fd)^2}\]

\[SNR = \frac{P_r}{N_0B}\]


Quanto maior d, menor \(P_r\), e quanto menor \(P_r\), menor SNR.
Quanto maior B, menor SNR, logo menor a eficiência.

\paragraph{2016R - 1}
\begin{itemize}
    \item baudrate (2B) = 80 kbaud
    \item bitrate (C) = 320 kbit/s
    \item phase modulation
    \item nº of phases = ?
\end{itemize}

\[C = 2B\log_2 M\]
\[320 = 80\log_2 M\]
\[4 = log_2 M\]
\[M = 2^4 = 16\]

\paragraph{2016N - 2}
\begin{itemize}
    \item 16 QAM
    \item bitrate (C) = 100kbit/s
    \item baudrate (2B) = ?
\end{itemize}

\[C = 2B\log_2 M\]
\[100 = 2B\log_2 16\]
\[100 = 2B*4\]
\[2B = 25\]

\paragraph{2016N - 3}
Canal rádio com propagação em espaço livre.

\[\frac{P_t}{P_r}=\frac{(4\lambda fd)^2}{c^2}\]
\[{P_r}=\frac{P_t}{\frac{(4\lambda fd)^2}{c^2}}\]
\[{P_r}=\frac{P_tc^2}{(4\lambda fd)^2}\]

\[SNR = \frac{P_r}{N_0B}\]
\[C = B_c\log_2 (1 + SNR)\]

Quanto menor a distância e frequência, maior \(P_r\).
Quanto maior \(P_r\), maior SNR, logo maior a capacidade.

\paragraph{2015 - 2}
\begin{itemize}
    \item 2 ligações sem fios
    \item Pt1 = Pt2 (potência transmitida pelo emissor)
    \item B1 = B2 (largura de banda do canal)
    \item d1 < d2 (distância entre o emissor e o recetor)
    \item relação entre P e C das ligações?
\end{itemize}

\[\frac{P_t}{P_r}=\frac{(4\lambda fd)^2}{c^2}\]
\[P_t=P_r\frac{(4\lambda fd)^2}{c^2}\]
De Pt1 = Pt2,
\[P_r1*(4\lambda fd_1)^2 = P_r2*(4\lambda fd_2)^2\]
Como d1 < d2, P\_r1 > P\_r2, então C1 > C2
\[C_1 = B_1\log_2 (1 + \frac{P_r1}{N_0B_1})\]
\[C_2 = B_1\log_2 (1 + \frac{P_r2}{N_0B_1})\]

\paragraph{2014N - 1}
\begin{itemize}
    \item baudrate (2B) = 8 kbaud
    \item bitrate (C) = 32 kbit/s
    \item bandwith (B) = 4 kHz
    \item M = ?
\end{itemize}

\[C = 2B\log_2 M\]
\[32 = 8\log_2 M\]
\[4 = \log_2 M\]
\[M = 16\]

\paragraph{2013N - 1}
\begin{itemize}
    \item bandwith (B) = 1 MHz
    \item baudrate (2B) = 2 MHz
    \item SNR = 40 dB
    \item 8 level impulses => M = 8
    \item bitrate (C) = ?
\end{itemize}

\[C = 2B\log_2 M\]
\[C = 2\log_2 8\]
\[C = 2*3 = 6\]

\paragraph{2012N - 1}
\begin{itemize}
    \item 4 QAM
    \item baudrate (2B) = 100 kbaud
    \item bitrate (C) = ?
\end{itemize}

\[C = 100\log_2 4\]
\[C = 100*2\]
\[C = 200\]

\paragraph{2011N - 2}
Num canal sem fios, potência recebida é tanto maior quanto menor for a distância emissor-recetor e o comprimento de onda
da portadora.

\[\frac{P_t}{P_r}=\frac{(4\lambda fd)^2}{c^2}\]
\[P_r=\frac{P_t}{\frac{(4\lambda fd)^2}{c^2}}\]
\[P_r=\frac{P_tc^2}{(4\lambda fd)^2}\]

\end{document}