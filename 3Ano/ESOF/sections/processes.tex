\documentclass[../ESOF_notes.tex]{subfiles}
 
\begin{document}

Software engineering - systematic, disciplined, quantifiable approach to the development, operation, and maintenance of software – cost-effective, timely, predictable, high-quality

Software process - structured set of activities required to develop a software system.

\paragraph{Following process activities}
\begin{itemize}
    \item Specification – defining what the system should do;
    \item Design and implementation – defining the organization of the system and implementing the system;
    \item Validation – checking that it does what the customer wants;
    \item Evolution – changing the system in response to changing customer needs.
\end{itemize}

\paragraph{Types of processes}
\begin{itemize}
    \item Plan-driven: planned in advance and progress is measured against this plan
    \item Agile: planning is incremental
    \item Most approaches nowadays combine both methods
\end{itemize}

\paragraph{Why define processes?}
\begin{itemize}
    \item Efficiency – helps to keep focus and structure
    \item Consistency - results likely to be similar
    \item Basis for Improvement - gathering data on your work -> room for improvement
\end{itemize}

\subsection{Process Activities}
\paragraph{Software specification (or requirements engineering)}
\begin{itemize}
    \item Requirements elicitation and analysis - What do the system stakeholders require or expect from the system?
    \item Requirements specification - Defining the requirements in detail
    \item Requirements validation - Checking the validity of the requirements
\end{itemize}

\paragraph{Software design and implementation}

Design - design a software structure that realises the specification

\begin{itemize}
    \item Architectural design – overall structure
    \item Database design – system data structure
    \item Interface design – between system components
    \item Component (or detailed) design – design of each component individually
\end{itemize}

Implementation - translate the design into an exec. prog.

\paragraph{Software verification and validation (testing)}
the system conforms to its specification (verification)
meets the requirements and customer needs (validation).

Testing

\begin{itemize}
    \item Unit (or component) testing – individual component testing
    \item Integration testing – testing of interaction between components
    \item System testing – general testing (performance, usability, etc…)
    \item Acceptance testing – live testing with data
\end{itemize}

\paragraph{Software evolution (or maintenance) – after development}
\begin{itemize}
    \item Corrective – bug fixing
    \item Adaptive – adapt to new platforms, technologies
    \item Perfective – new functionalities
\end{itemize}

\subsection{Software process models}
Waterfall model (plan-driven) – Separate specification and development

\begin{itemize}
    \item Inflexible partitioning of the project – hard to respond to changing requirements
    \item Used for large systems engineering projects where a system is developed at several sites – plan-driven aspect helps with coordination
\end{itemize}

Incremental development (\& delivery) (agile or plan-driven) - Specification, development and validation are interleaved.

\begin{itemize}
    \item Easier to adapt to changing requirements
    \item More feedback – reduced risk of failure
    \item Can be delivered staggered
    \item Needs constant refactoring (due to the multiple increments)
    \item Suboptimal reusability
\end{itemize}
Integration and configuration (agile or plan-driven) - The system is assembled from existing configurable components.

\begin{itemize}
    \item Reduced costs and risks – less software developed from scratch
    \item Faster system delivery
    \item Needs requirement compromises to fit existing components
    \item Loss of control over the evolution of the used components
\end{itemize}

Software prototyping - Not actually a model but an approach to cope with uncertainty
\begin{itemize}
    \item A prototype is an initial version of a system used to demonstrate concepts and try out design options – reduced uncertainty
\end{itemize}

\subsection{RUP - Rational Unified Process}

\subsubsection{Best Practices}
\begin{itemize}
    \item Develop iteratevely
    \item Manage requirements
    \item Use component architectures
    \item Model visually (UML)
    \item Continuously verify quality
    \item Manage change
\end{itemize}

\subsubsection{Phases}

Each phase has several iterations, that walk through all disciplines.

\begin{itemize}
    \item Inception
    
    Define the project scope (understand the problem)
    \item Elaboration
    
    Define the solution architecture (understand the solution)
    \item Construction
    
    Build the product
    \item Transition
    
    Transition the product into the end-users
\end{itemize}

\subsubsection{Disciplines}

\begin{itemize}
    \item Business modeling
    \item Requirements
    \item Analysis \& design
    \item Implementation
    \item Test
    \item Deployment
    \item Configuration \& change management
    \item Project management
    \item Environment
\end{itemize}

\subsubsection{Basic elements}
\begin{itemize}
    \item Role
    \item Activity
    \item Artifact
    \item Workflow \& Workflow Detail
\end{itemize}

\end{document}