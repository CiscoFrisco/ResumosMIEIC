\documentclass[../ESOF_notes.tex]{subfiles} 
 
\begin{document} 

\paragraph{What is it?}

The process of studying customer and user needs to arrive at a definition of system, hardware, or software requirements (i.e., a property that the software must have)

\paragraph{Importance} 
Many defects can be traced to the (requirements) specification. These can be very hard to fix, since they are very structural to the project and, if discovered late, may be spread throughout the project

\paragraph{Problems}

(mis)communication \& (mis)understanding

Evolving requirements – requirements creep: uncontrolled changes or continuous growth in a project’s requirements

\paragraph{Levels of software requirements}
\begin{itemize}
    \item Business requirements/needs - high-level objectives
    vision and scope document
    \item User requirements/needs- goals or tasks that the user must be able to perform with the product
    use case models
    \item System requirements – requirements for the system as a whole (HW/SW)
    system requirements specification document
    \item Software requirements – derived from system requirements
    software requirements specification (SRS) document
\end{itemize}

\paragraph{Types of software requirements}
Functional requirements (capabilities) - functions that the software is to execute

Nonfunctional requirements - act to constrain the solution
\begin{itemize}
    \item Mostly quality requirements – Example: The maximum system down-time should be 8 hours per year
    \item Can also include development process requirements (such as programming languages, etc.)
\end{itemize}

Quality requirements - quality characteristics, sub-characteristics and metrics

\paragraph{ISO/IEC 25010 standard}
    \begin{itemize}
        \item Functionality suitability – degree to which provides functions that meet stated and implied needs
        \item Performance efficiency – performance relative to execution conditions
        \item Reliability – degree to which specified functions are performed under specified conditions for a specified period of time
        \item Usability - effectiveness, efficiency and satisfaction in a specified context of use
        \item Compatibility - degree to which information can be exchanged with other products, systems or components
        \item Maintainability – degree of effectiveness and efficiency during modification
        \item Portability - degree of effectiveness and efficiency during hardware or software transfer
        \item Security – degree of information and data protection
    \end{itemize}

\paragraph{Requirements Engineering and Agile processes}
Requirements Engineering mainly goes against the agile methodology to not overplan before development

Solution: user stories - Lightweight way to record a software need, with just enough information

\subsection{Requirements engineering process}

\paragraph{Requirements elicitation (or discovery)}

Interact with stakeholders and other sources (documents, existing systems, etc.) to collect/discover their requirements

\paragraph{Problems}

\begin{itemize}
    \item Problems of scope – ill-defined boundaries, unnecessary information
    \item Problems of understanding
    \item Problems of volatility - Requirements evolve over time.
\end{itemize}

\paragraph{Requirements analysis (\& negotiation)}

Detect and resolve problems with the requirements

Checklist
\begin{itemize}
    \item Completeness
    \item Consistency
    \item Unambiguity
    \item Verifiability
    \item Necessity
    \item Feasibility
\end{itemize}

\paragraph{Requirements specification}
Production a Software Requirements Specification (SRS) document, as well as other docs, Prototypes, Models, etc..

\paragraph{Requirements validation}

Demonstrate that the requirements define the system that the customer really wants. Attempts to prevent the costly errors (as said before) that can happen during requirements

\subsection{Requirements elicitation techniques}

\paragraph{Interviews - Most widely used requirements elicitation technique}
\begin{itemize}
    \item Open/unstructured - various issues are explored with stakeholders
    \item Closed/structured - based on a pre-determined list of questions
    \item Mixed
\end{itemize}

\paragraph{Brainstorming - Useful to elicit new and innovative requirements}

\begin{itemize}
    \item Moderator and 4-8 people
    \item Generate new ideas and discuss, review and organize them
\end{itemize}
Questionnaires (surveys) - Well-suited for confirming/prioritizing previously identified candidate requirements
Set of questions


Goal analysis - Hierarchical decomposition of stakeholder goals to derive system requirements

Social Observation and Analysis - Requirements can be derived from the external observation of the routine way and tactics of work

Prototyping - initial/primitive version of a system (cheaper, faster to develop, limited in functionality)
\begin{itemize}
    \item Throw-away prototypes - focus on requirements rather than implementation constraints
    \item Evolutionary prototypes - Appropriate for rapid, iterative, application development
\end{itemize}


\subsection{System models in requirements engineering}

A simplified representation of a system (as-is or to-be) from a certain perspective - tackle complexity through abstraction. 
Also helps removing the ambiguity and lack of structure inherent to natural language descriptions.
In requirements engineering:
\begin{itemize}
    \item Use case model - for organizing functional requirements
    \item Domain model – for organizing the vocabulary and information requirements
\end{itemize}


\subsubsection{Use case model}
Use case diagram(s) + associated documentation

Purpose: show the system purpose and usefulness, capture functional requirements (through the use cases), specify the system context (actors)

Actors
\begin{itemize}
    \item user role or external system
\end{itemize}

Use cases
\begin{itemize}
    \item Functionality or service as perceived by users
    \item Type of interaction between actors and the system
    \item Sequence of actions, including variants, resulting in an observable
    result with value for an actor
    
\end{itemize}

Relationships

\begin{itemize}
    \item Generalization
    \item Extension
    \item Include
\end{itemize}

\subsubsection{Domain Modeling}
\begin{itemize}
    \item Used to organize the vocabulary of the problem domain or to capture information requirements
    \item Represented through UML class diagrams
    \item Can use integrity constraints (or invariants) associated with classes to restrict valid object states
\end{itemize}

\end{document}