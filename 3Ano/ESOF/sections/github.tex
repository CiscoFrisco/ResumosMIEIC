\documentclass[../ESOF_notes.tex]{subfiles}
 
\begin{document} 

\subsection{Importance of Issues and Pull Requests}

Before GitHub, open source project developers could receive feedback from multiple means, e.g. forums, e-mail, chat, etc. which means there wasn't a clear way to find and contribute to existing problems.

Issues and Pull Requests come to fix this, by centralizing everything. Take for example this year's third project, where students had to learn how to contribute to open-source projects on GitHub.

The Issues and Pull Requests tabs are incredibly useful since they agreggate the projects current and past issues and the decision making that goes into fixing them, making it easier to contribute to them.
This way, it's also easier to find current problems, thus denying the possibility of duplicate threads.

\subsection{ChatOps}

Concept introduced by a GitHub employee. In an example shown in the talk, GitHub has a Slack workspace where they have deployed a bot that can do almost everything they need.
From showing cat images to displaying informations about employees, repositories, pull requests, etc.

\subsection{Advice in Engineering}
\begin{itemize}
    \item Shift left - perform testing earlier in the lifecycle
    \item Pattern importance
    \item Readable code \& architecture
    \item Effective technical communication
    \item Re-use more / invent less
    \item Automate everything
    \item Deploy, measure, improve
    \item Early on, prioritise learning
    \item Find ways to constantly be challenged
    \item Understand selling
\end{itemize}

\end{document}